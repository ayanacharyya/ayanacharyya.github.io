%%%%%%%%%%%%%%%%%%%%%%%%%%%%%%%%%%%%%%%%%%%%%%%%%%%%%%%%%%%%%%%%%%%%%%%%
%%%%%%%%%%%%%%%%%%%%%% Simple LaTeX CV Template %%%%%%%%%%%%%%%%%%%%%%%%
%%%%%%%%%%%%%%%%%%%%%%%%%%%%%%%%%%%%%%%%%%%%%%%%%%%%%%%%%%%%%%%%%%%%%%%%

%%%%%%%%%%%%%%%%%%%%%%%%%%%%%%%%%%%%%%%%%%%%%%%%%%%%%%%%%%%%%%%%%%%%%%%%
%% NOTE: If you find that it says                                     %%
%%                                                                    %%
%%                           1 of ??                                  %%
%%                                                                    %%
%% at the bottom of your first page, this means that the AUX file     %%
%% was not available when you ran LaTeX on this source. Simply RERUN  %%
%% LaTeX to get the ``??'' replaced with the number of the last page  %%
%% of the document. The AUX file will be generated on the first run   %%
%% of LaTeX and used on the second run to fill in all of the          %%
%% references.                                                        %%
%%%%%%%%%%%%%%%%%%%%%%%%%%%%%%%%%%%%%%%%%%%%%%%%%%%%%%%%%%%%%%%%%%%%%%%%

%%%%%%%%%%%%%%%%%%%%%%%%%%%% Document Setup %%%%%%%%%%%%%%%%%%%%%%%%%%%%

% Don't like 10pt? Try 11pt or 12pt
\documentclass[10pt]{article}

% This is a helpful package that puts math inside length specifications
\usepackage{calc}
\usepackage{pifont}
\usepackage{marvosym}
\usepackage{amsmath}	% Advanced maths commands
\usepackage{amssymb}	% Extra maths symbols

% Simpler bibsection for CV sections
% (thanks to natbib for inspiration)
\makeatletter
\newlength{\bibhang}
\setlength{\bibhang}{1em}
\newlength{\bibsep}
{\@listi \global\bibsep\itemsep \global\advance\bibsep by\parsep}
\newenvironment{bibsection}%
{\vspace{-\baselineskip}\begin{list}{}{%
			\setlength{\leftmargin}{\bibhang}%
			\setlength{\itemindent}{-\leftmargin}%
			\setlength{\itemsep}{\bibsep}%
			\setlength{\parsep}{\z@}%
			\setlength{\partopsep}{0pt}%
			\setlength{\topsep}{0pt}}}
	{\end{list}\vspace{-.6\baselineskip}}
\makeatother

% Layout: Puts the section titles on left side of page
\reversemarginpar

%
%         PAPER SIZE, PAGE NUMBER, AND DOCUMENT LAYOUT NOTES:
%
% The next \usepackage line changes the layout for CV style section
% headings as marginal notes. It also sets up the paper size as either
% letter or A4. By default, letter was used. If A4 paper is desired,
% comment out the letterpaper lines and uncomment the a4paper lines.
%
% As you can see, the margin widths and section title widths can be
% easily adjusted.
%
% ALSO: Notice that the includefoot option can be commented OUT in order
% to put the PAGE NUMBER *IN* the bottom margin. This will make the
% effective text area larger.
%
% IF YOU WISH TO REMOVE THE ``of LASTPAGE'' next to each page number,
% see the note about the +LP and -LP lines below. Comment out the +LP
% and uncomment the -LP.
%
% IF YOU WISH TO REMOVE PAGE NUMBERS, be sure that the includefoot line
% is uncommented and ALSO uncomment the \pagestyle{empty} a few lines
% below.
%

%% Use these lines for letter-sized paper
%\usepackage[paper=letterpaper,
%           %includefoot, % Uncomment to put page number above margin
%            marginparwidth=0.7in,     % Length of section titles
%            marginparsep=.05in,       % Space between titles and text
%            margin=0.5in,               % 1 inch margins
%            includemp]{geometry}

% Use these lines for A4-sized paper
\usepackage[paper=a4paper,
%includefoot, % Uncomment to put page number above margin
marginparwidth=24mm,    % Length of section titles
marginparsep=1mm,       % Space between titles and text
margin=15mm,              % 25mm margins
includemp]{geometry}

%% More layout: Get rid of indenting throughout entire document
\setlength{\parindent}{0in}

%% This gives us fun enumeration environments. compactitem will be nice.
\usepackage{paralist}
\usepackage{etaremune}
\usepackage[shortlabels]{enumitem}
% \usepackage[misc]{ifsym}
%% Reference the last page in the page number
%
% NOTE: comment the +LP line and uncomment the -LP line to have page
%       numbers without the ``of ##'' last page reference)
%
% NOTE: uncomment the \pagestyle{empty} line to get rid of all page
%       numbers (make sure includefoot is commented out above)
%
\usepackage{fancyhdr,lastpage}
\pagestyle{fancy}
%\pagestyle{empty}      % Uncomment this to get rid of page numbers
\fancyhf{}\renewcommand{\headrulewidth}{0pt}
\fancyfootoffset{\marginparsep+\marginparwidth}
\newlength{\footpageshift}
\setlength{\footpageshift}
{0.1\textwidth+0.1\marginparsep+0.1\marginparwidth-2in}
\lfoot{\hspace{\footpageshift}%
	\parbox{3.5in}{\, \hfill %
		\arabic{page} of \protect\pageref*{LastPage} % +LP
		%                    \arabic{page}                               % -LP
		\hfill \,}}

% Finally, give us PDF bookmarks
\usepackage{color,hyperref}
\definecolor{darkblue}{rgb}{0.0,0.0,0.3}
\hypersetup{colorlinks,breaklinks,
	linkcolor=darkblue,urlcolor=darkblue,
	anchorcolor=darkblue,citecolor=darkblue}

%%%%%%%%%%%%%%%%%%%%%%%% End Document Setup %%%%%%%%%%%%%%%%%%%%%%%%%%%%


%%%%%%%%%%%%%%%%%%%%%%%%%%% Helper Commands %%%%%%%%%%%%%%%%%%%%%%%%%%%%

% The title (name) with a horizontal rule under it
%
% Usage: \makeheading{name}
%
% Place at top of document. It should be the first thing.
\newcommand{\makeheading}[1]%
{\hspace*{-\marginparsep minus \marginparwidth}%
	\begin{minipage}[t]{\textwidth+\marginparwidth+\marginparsep}%
		{\large \bfseries #1}\\[-0.15\baselineskip]%
		\rule{\columnwidth}{1pt}%
\end{minipage}}

% The section headings
%
% Usage: \section{section name}
%
% Follow this section IMMEDIATELY with the first line of the section
% text. Do not put whitespace in between. That is, do this:
%
%       \section{My Information}
%       Here is my information.
%
% and NOT this:
%
%       \section{My Information}
%
%       Here is my information.
%
% Otherwise the top of the section header will not line up with the top
% of the section. Of course, using a single comment character (%) on
% empty lines allows for the function of the first example with the
% readability of the second example.
\renewcommand{\section}[2]%
{\pagebreak[2]\vspace{1\baselineskip}%
	\phantomsection\addcontentsline{toc}{section}{#1}%
	\hspace{0in}%
	\marginpar{
		\raggedright \scshape #1}#2}

% An itemize-style list with lots of space between items
\newenvironment{outerlist}[1][\enskip\textbullet]%
{\begin{itemize}[#1]}{\end{itemize}%
	\vspace{-0.6\baselineskip}}

% An environment IDENTICAL to outerlist that has better pre-list spacing
% when used as the first thing in a \section
\newenvironment{lonelist}[1][\enskip\textbullet]%
{\vspace{-\baselineskip}\begin{list}{#1}{%
			\setlength{\partopsep}{0pt}%
			\setlength{\topsep}{0pt}}}
	{\end{list}\vspace{-.6\baselineskip}}

% An itemize-style list with little space between items
% \newenvironment{innerlist}[1][\enskip\textbullet]%
\newenvironment{innerlist}[1][\enskip$\circ$]%
{\begin{compactitem}[#1]}{\end{compactitem}}

% An environment IDENTICAL to innerlist that has better pre-list spacing
% when used as the first thing in a \section
\newenvironment{loneinnerlist}[1][\enskip\textbullet]%
{\vspace{-\baselineskip}\begin{compactitem}[#1]}
	{\end{compactitem}\vspace{-.6\baselineskip}}

% To add some paragraph space between lines.
% This also tells LaTeX to preferably break a page on one of these gaps
% if there is a needed pagebreak nearby.
\newcommand{\blankline}{\quad\pagebreak[2]}

% Uses hyperref to link DOI
\newcommand\doilink[1]{\href{http://dx.doi.org/#1}{#1}}
\newcommand\doi[1]{doi:\doilink{#1}}

% These are for references to work properly
\def\aj{AJ}                  
\def\apj{ApJ}                  
\def\apjl{ApJL}    
\def\mnras{MNRAS}                  
\def\araa{ARA\&A}
\def\pasp{PASP}
\def\apjs{ApJS}  
\def\aap{A\&A}  
\def\aaps{A\&AS}  
\def\apss{Ap\&SS}
\def\nat{Nature}
\def\pasa{PASA}
\def\aapr{AARv}


%%%%%%%%%%%%%%%%%%%%%%%% End Helper Commands %%%%%%%%%%%%%%%%%%%%%%%%%%%

%%%%%%%%%%%%%%%%%%%%%%%%% Begin CV Document %%%%%%%%%%%%%%%%%%%%%%%%%%%%

%\hyphenpenalty = 9999
\def\vs{\vspace{-0.1in}}
\begin{document}
	% \makeheading{Curriculum Vitae\\ [0.3cm] TIEP HUU VU\quad~~~~~~\quad\quad\quad\quad\quad\quad\quad\quad\quad\quad\quad\quad\quad\quad{\small Last update: December 17, 2015}}
	\makeheading{Ayan Acharyya}
	
	
	\section{Contact Information}
	
	\newlength{\rcollength}\setlength{\rcollength}{3 in}
	\vs
	\begin{tabular}[t]{@{}p{\textwidth-\rcollength}p{\rcollength}}
		
		INAF--Padua Astronomical Observatory & \texttt{Homepage:}\href{https://ayanacharyya.github.io/}{https://ayanacharyya.github.io/}\\
		Vicolo Observatory 5,     &  {\large\Letter} \texttt{E-mail:}\href{mailto:aachary9@jhu.edu}{aachary9@jhu.edu}\\
		Padua 35122,     & \texttt{Tel}: (+35) 351-429-4809 \\
		Italy & Github: \href{https://github.com/ayanacharyya}{https://github.com/ayanacharyya}
	\end{tabular}
	
	%% ==============================================================
	\vspace{0.2in}
	\section{Research Expertise} % (fold)
	%\vspace{-0.27in}
	Galaxy evolution, Chemical evolution - gas phase metallicity, ISM properties, translating simulations to mock IFU datacubes, analysing cosmological hydrodynamical simulations.
	
	%% =========  ==============================
	\section{Post-PhD Experience} % (fold)
	\textbf{INAF Padova}, Italy. \hfill May 2024--present\\
	\textit{Post-doctoral researcher:} working on JWST/NIRISS data with the JWST-PASSAGE team.
	\vspace{0.1in}
	
	\textbf{Johns Hopkins University}, Baltimore, USA. \hfill January 2021--April 2024\\
	\textit{Assistant Research Scientist:} Post-doctoral researcher with the \href{https://foggie.science/who-we-are.html}{FOGGIE} group. This involves using Enzo to produce cosmological zoom-in simulations of galaxies and developing my own tools (\href{https://github.com/foggie-sims/foggie/tree/master/foggie/ayan_acharyya_codes}{Github link}) for creating mock data products.
	\vspace{0.1in}
	
	%% =========  ==============================
	\section{Education}
	\textbf{Australian National University}, Canberra, Australia; \hfill September 2015--September 2020\\
	\textbf{PhD}
	\begin{outerlist}
		\itemsep0em % to reduce space between bullet points
		\item Thesis title: \textit{Chemical evolution of the Universe across the cosmic time}
		\item Advisors: Prof. Lisa Kewley, Prof. Mark Krumholz, A/Prof. Christoph Federrath. 
	\end{outerlist}
	
	\vspace{0.2in}
	\textbf{Indian Institute of Technology Kharagpur}, India \hfill August 2010-- April 2015\\
	\textbf{Integrated Bachelors and Masters of Science}
	\begin{outerlist}
		\itemsep0em % to reduce space between bullet points
		\item Thesis title: \textit{Simulating HII bubble around quasars to be used for matched filter technique in redshifted 21cm maps}
		\item Advisor: Prof. Somnath Bharadwaj
	\end{outerlist}
	
	\vspace{0.2in}
	\textbf{University of Manitoba}, Winnipeg, Canada \hfill May--July 2014\\
	\textbf{MITACS Research Scholar}
	\begin{outerlist}
		\itemsep0em % to reduce space between bullet points
		\item Project title \textit{Colorizing the dance of galaxies}
		\item Advisor: Dr. Jayanne English
	\end{outerlist}
	
	%% =========  ==============================
	\section{Observing Experience} % (fold)
	\vspace{-0.33in}
	
	\begin{outerlist}
		\itemsep0em % to reduce space between bullet points
		\item 6 nights total on Keck/ESI, from Keck HQ at Waimea, Hawaii. I was co-I on two out of the three observing proposals.
		\item 1 night on ANU 2.3m telescope: WiFeS spectrograph.
	\end{outerlist}
	
	%% =========  ==============================
	\section{Successful Proposals} % (fold)
	\vspace{-0.33in}
	
	\begin{outerlist}
		\itemsep0em % to reduce space between bullet points
		\item \textit{How do galaxies move their metals? Unraveling z > 2 disks with controlled numerical experiments tailored to JWST}, JWST (Cycle 3), Theory/AR, \textbf{PI: Dr. Ayan Acharyya}
		\item \textit{CO Kinematics at Cosmic Noon: Timing the Redistribution of Metals Around Galaxies}, ALMA/Band3 (Cycle 8), PI: Dr. Raymond Simons
		\item \textit{Unwrapping the epoch of reionization through analogs at cosmic noon}, VLT/XSHOOTER (Cycle P108), PI: Dr. Anshu Gupta
		\item \textit{Rest-frame Ultraviolet spectroscopy of Two Lensed Galaxies at z=1.4}, Keck/ESI (2016B), PI: Dr. Fuyan Bian 
		\item \textit{Galaxy Feedback in two lensed galaxies at z=1.4}, Keck/ESI (2016B), PI: Dr. Jane Rigby 
		
	\end{outerlist}
	
	%% =========  ==============================
	\section{Highlights} % (fold)
	\vspace{-0.33in}
	
	\begin{outerlist}
		\itemsep0em % to reduce space between bullet points
		\item \textbf{Three first-authored} and 15 total \textbf{refereed publications}, with over 400 citations, and $h$-index = 10.
		\item Given over \textbf{30 talks and colloquia} at conferences and institutes across six countries, including four invited talks. Hosted over 30 outreach tours, involving public talks.
		\item Secured over \textbf{\$12,000 in awards} and travel grants.
		\item Total \textbf{7 nights observing experience} on Keck telescope in Hawaii and the ANU 2.3m telescope in Australia.
		\item \textbf{PI on one} successful JWST theory proposal and \textbf{Co-I on four} successful observing proposals.
		\item \textbf{Mentor/research-supervisor} for two high-school students and one undergraduate student.
		\item \textbf{Co-organised four conferences}, including an international one.
	\end{outerlist}
	
	% =========  ==============================
	\section{Awards and Grants} % (fold)
	\vspace{-0.32in}
	\begin{etaremune}
		\itemsep0em % to reduce space between bullet points
		\item 2023: AAS International Travel Grant \$1700
		\item 2023: Astro3D visitor travel grant \$3500
		\item 2019: RSAA student travel grant \$4000
		\item 2019: Astronomical Society of Australia (ASA) student travel award \$1000
		\item 2019: ANU Vice Chancellor's travel grant \$1500
		\item 2017: Olin J Eggen Research Award 2017 at RSAA, ANU
		\item 2015: ANU PhD Scholarship (International) and RSAA Research Supplementary Scholarship
		\item 2014: MITACS Globalink Research Internship award
		% \item 2014: Visiting Students Programme at Tata Institute of Funamental Research(TIFR) Mumbai, India (declined)
		% \item 2014: NCTU Elite Internship Programme, Taiwan (declined)
		% \item 2014: Charpak Fellowship for summer project in France (declined)
		\item 2013: Visiting Students Research fellowship (Indian Institute of Technology Gandhinagar)
		\item Second-best poster award in the Theme Meeting on Ultrafast Science UFS 2013, IIT Kharagpur
		\item 2012: Visiting Students Research fellowship (Indian Academy of Sciences)
	\end{etaremune}
	
	%% =========  ==============================
	\section{Mentoring/Teaching Experience} % (fold)
	%\vspace{-0.25in}
	\begin{etaremune}
		\item \textbf{Summer At Space telescope Program (SASP) mentor}: I was the primary mentor of an undergraduate student for an intensive 9 week summer research program, as part of the SASP program, which is aimed at providing a diverse, international batch of students with an opportunity of a short research experience. As primary mentor my role ranged from guiding my student on broad science research to teaching and helping them learn soft skills such as coding, writing and science communication. Their work led to a non-refereed publication on RNAAS. \hfill \textit{Baltimore, USA;} June - August 2023
		\item \textbf{Exploring the Universe with Space Telescopes}: I was one of the two competitively hired course instructors for this 2 week intensive summer course at Johns Hopkins University for prospective undergraduate students. My role included delivering lectures, evaluating tests, and manage the cohort of 20-30 students. \hfill \textit{Baltimore, USA;} July - August 2023
		\item \textbf{One Sky}: Volunteered as one of the two instructors for a fully-virtual, free, 18 week Introduction to Astronomy course for high school students in India, as a part of \href{https://www.openfieldcollective.org/astronomy}{Open Field Collective}. I was responsible for developing the course materials for student sin Grade 8-10, as well as delivering the lecture in weekly 1-2 hour sessions. \hfill \textit{Virtual;} August - December 2022
		\item \textbf{Science Mentors ACT}: Mentored two high-school students for their `ACT Science Mentors' Project, on "Cepheid Variables" and "Eclipsing binaries" respectively, on the MSATT telescope at Mount Stromlo Observatory. I was responsible for teaching them the relevant physics and mathematics as well as help them with the data analysis and report writing.
		\vspace{-10pt}
		\begin{flushright}
			\textit{Canberra, Australia;} August 2018 - March 2019
		\end{flushright}
	\end{etaremune}
	
	%% =========  ==============================
	\section{Service} % (fold)
	\vspace{-0.32in}
	\begin{etaremune}
		\itemsep0em % to reduce space between bullet points
		\item Co-organiser of the weekly \textbf{Informal Science Hour} at Space Telescope Science Institute. I initiated this meeting format to revive the in-person, collaborative and scientific environment at STScI post-pandemic, and it has so far been very successful. My role includes soliciting discussion leaders every week, run the meetings, and organising some light refreshments (sponsored by the institute). \hfill December 2022 - current
		\item \textbf{Panel Support Scientist for the James Webb Space Telescope} Cycle 2 Time Allocation Committee. This involved all day virtual meetings for a week, to help smooth running of one of the JWST panels that met virtually to discuss and rank telescope proposals. My role was to manage the grading software platform, handle queries from the panelists, and assist the panel chair in running the meeting. \hfill April 2023
		\item Over 30 stargazing tours as \textbf{Outreach Assistant} at \textbf{Mount Stromlo Observatory outreach team} \hfill 2017--2020
		\item Organiser of \textbf{GEARS3D group meeting} at RSAA \hfill 2018--2020
		\item OC member of the \textbf{ASTRO3D Student Retreat} \hfill May 2019
		\item LOC member of the \textbf{Harley Wood School of Astronomy} \hfill July 2017
		\item PhD student representative on the \textbf{RSAA Education Committee} \hfill June 2016 - February 2017
		\item LOC/SOC member of the \textbf{Mount Stromlo Student Seminars} \hfill December 2016
		\item LOC member of the \textbf{DAE-BRNS Theme meeting on Ultrafast Science}, Kharagpur \hfill 2013
	\end{etaremune}
	
	%% =========  ==============================
	\section{Talks\\\textit{Conferences\\(Contributed talks)}} % (fold)
	\vspace{-0.25in}
	\begin{etaremune}
		\itemsep0em % to reduce space between bullet points
		\item[]
		\item \textbf{STScI Spring Symposium: Recipes to Regulate Star Formation at All Scales} Robust measurements of gas-phase metallicity distributions with FOGGIE simulations \hfill \textit{Baltimore, USA;} April 2024
		\item \textbf{Space Telescope Science Institute, Hot-Sci Colloquium series} Gas-phase metallicity distributions with FOGGIE simulations \hfill \textit{Baltimore, USA;} August 2023
		\item \textbf{Feedback \& the Baryon Cycle in Galaxies} Gas-phase metallicity distributions with FOGGIE simulations \hfill \textit{Healesville, Australia;} July 2023
		\item \textbf{Oases in the Cosmic Desert Conference} Robust measurements of gas-phase metallicity distributions with FOGGIE simulations \hfill \textit{Tempe, USA;} February 2023
		\item \textbf{Space Telescope Science Institute, Discovery Seminar series} “Mockulus reparo” –- to fix the effects on metallicity gradient measurements due to our insufficient “seeing” [\href{https://www.youtube.com/watch?v=atXtjtjrbDo}{Recording link}] \hfill \textit{Baltimore, USA;} May 2022 
		\item \textbf{Johns Hopkins University} “Mockulus reparo” –- to fix the effects on metallicity gradient measurements due to our insufficient “seeing” \hfill \textit{Baltimore, USA;} September 2021
		\item \textbf{Chemical Abundances in Gaseous Nebulae} "Abundances  from  UV  spectra  at  high-redshift" \hfill \textit{virtual;} May 2021
		\item \textbf{American Astronomical Society (AAS) 2019} "Testing new rest-frame optical \& UV diagnostics on lensed galaxy at z$\sim$1.7" \hfill \textit{Seattle, USA;} January 2019
		\item \textbf{AAS 2019} "Determining effects of telescope resolution on metallicity gradient with synthetic observations of galaxy simulations" \hfill \textit{Seattle, USA;} January 2019
		\item \textbf{Australian National Institute for Theoretical Astrophysics (ANITA)} \textit{Perth, Australia;} February 2018
		\item \textbf{5th Annual GMT Community Science Meeting} \textit{New York, USA;} July 2017
		\item \textbf{ASA Annual Science Meeting} \hfill \textit{Canberra, Australia;} July 2017
		\item \textbf{Mount Stromlo Student Seminars} \hfill \textit{Canberra, Australia;} December 2015
	\end{etaremune}
	
	\section{\textit{Colloquia}} % (fold)
	\vspace{-0.25in}
	\begin{etaremune}
		\itemsep0em % to reduce space between bullet points
		\item \textbf{INAF Padova} (Invited) \textit{Padova, Italy;} April 2024
		\item \textbf{Macquarie University} (Contributed) \hfill \textit{Sydney, Australia;} July 2023
		\item \textbf{University of New South Wales} (Contributed) \hfill \textit{Sydney, Australia;} July 2023
		\item \textbf{Australian National University} (Contributed) \hfill \textit{Canberra, Australia;} July 2023
		\item \textbf{Curtin University} (Contributed) \hfill \textit{Perth, Australia;} June 2023
		\item \textbf{University of Western Australia} (Contributed) \hfill \textit{Perth, Australia;} June 2023
		
		\item \textbf{University of Connecticut} (Invited) \hfill \textit{Hartford, USA;} March 2023
		\item \textbf{Universidad Nacional Autonoma de Mexico} (Contributed) \textit{Mexico City;} September 2019
		\item \textbf{University of Texas at Austin} (Contributed) \hfill \textit{Austin, USA;} September 2019
		\item \textbf{Ohio State University} (Contributed) \hfill \textit{Columbus, USA;} September 2019
		\item \textbf{New York University} (Contributed) \hfill \textit{New York City, USA;} September 2019
		\item \textbf{Space Telescope Science Institute} (Contributed)\hfill  \textit{Baltimore, USA;} September 2019
		\item \textbf{Sri Venkateswara College of Engineering} (Invited) \hfill \textit{Chennai, India;} March 2019
		\item \textbf{Vellore Institute fo Technology} (Invited) \hfill \textit{Vellore, India;} March 2019
		\item \textbf{R V College of Engineering} (Invited) \hfill \textit{Bengaluru, India;} March 2019
		\item \textbf{Leiden Observatory} (Contributed) \hfill \textit{Leiden, Netherlands;} September 2018
		\item \textbf{Max Planck Institute for Astronomy} (Contributed) \textit{Heidelberg, Germany;} September 2018
		\item \textbf{Institute for Theoretical Astrophysics} (Contributed) \textit{Heidelberg, Germany;} September 2018
		\item \textbf{Indian Institute of Technology} (Contributed) \hfill \textit{Kharagpur, India;} December 2016
		\item \textbf{National Centre for Radio Astrophysics} (Contributed) \hfill \textit{Pune, India;} December 2016
	\end{etaremune}
	
	\section{\textit{Outreach}} % (fold)
	\vspace{-0.25in}
	\begin{etaremune}
		\itemsep0em % to reduce space between bullet points
		\item \textbf{Mount Stromlo Observatory Space Squad} (Invited) \hfill  \textit{Canberra, Australia;} April 2019
		\item \textbf{Physics in the Pub} (Invited) \hfill \textit{Canberra, Australia;} October 2018
	\end{etaremune}
	
	\section{\textit{Posters}} % (fold)
	\vspace{-0.25in}
	\begin{etaremune}
		\itemsep0em % to reduce space between bullet points
		\item \textbf{EAS 2024 S12: Zooming In, Zooming Out: Exploring Galaxy Formation through Simulations} \hfill \textit{Padova, Italy;} July 2024
		\item \textbf{GAS-PISA Conference} \hfill \textit{Pisa, Italy;} May 2024
		\item \textbf{JWST First Science Conference} \hfill \textit{Baltimore, USA;} December 2022
		\item \textbf{IAU Focus Meeting} \hfill \textit{Vienna, Austria;} August 2018
		\item \textbf{ASA Annual Science Meeting} \hfill \textit{Melbourne, Australia;} July 2018
		\item \textbf{DAE-BRNS Theme Meeting on Ultrafast Science} \hfill \textit{Kharagpur, India;} 2013
	\end{etaremune}
	
	%% =========  ==============================
	\section{Pre-PhD Research Experience} % (fold)
	\vspace{-0.155in}
	
	\textbf{University of Manitoba}, Winnipeg, Canada. \hfill May--July 2014\\
	\textit{MITACS Research Scholar:} Project title ``Colorizing the dance of galaxies'' with Dr. Jayanne English. This involved simulating galaxies spanning diverse morphologies with a MATLAB based code `Ferret'.
	\vspace{0.1in}
	
	\textbf{Indian Institute of Technology Gandhinagar}, India. \hfill May--July 2013\\
	\textit{Summer Research Scholar:} ``Black Hole Kinematic'' with Dr. Sudipta Sarkar. I used Mathematica to investigate the evolution of the event horizon of a  Schwarzschild Black Hole under small perturbations in the mass.
	\vspace{0.2in}
	
	\textbf{Indian Institute of Technology Kharagpur}, India. \hfill January--April 2013\\
	\textit{Summer Research Scholar:} ``Z Scan based non linear optical characterization of nano-materials'' with Prof. Prasanta K. Datta.
	\vspace{0.2in}
	
	\textbf{Bhabha Atomic Research Centre}, Mumbai, India. \hfill May--July 2012\\
	\textit{Summer Research Scholar:} ``Small Angle Neutron Scattering Studies of Biological Systems in Solution'' with Dr. Vinod K. Aswal.
	\vspace{0.2in}
	
	% =============== Publications ===============================================
	\section{Publications\\\textit{First-authored}}
	\vspace{-0.1in}
	\input{firstauthored}
	
	\section{\textit{Co-authored}} % (fold)
	\vspace{-0.25in}
	\input{coauthored}
	
	% =============== Referees ===============================================
	\section{References}
	%\vspace{-0.2in}
	\input{referee_list}
	
\end{document}